\area{Formal Education}

\entry{Bachelor's Programme in Physics, Stockholms Universitet}
      {2011 -- Ongoing}
      {I take part in the extended programme 'Forskargrenen' which amounts to
       studying at a higher pace and approching more advanced areas earlier
       in the programme. The programme focuses on theory, problem solving, logical and analytical
       thinking, numeracy, planning and performing experiments,
       numerical analysis (mainly with MATLAB) and the ability to communicate
       complex ideas.}

\entry{Additional courses in Mathematics, Stockholm University}
      {2011 -- Ongoing}
      {My interest for mathematics is no less than that for physics.
       In addition to the courses (60 ECTS points) included in the physics
       programme I have taken courses in \emph{Ordinary Differential Equations},
       \emph{Foundations of Analysis}, \emph{Mathematical Logic} and
       \emph{Algebra and Combinatorics}.}

\entry{Additional courses in Computer Science, Various Universities}
      {2011 -- Ongoing}
      {Formal education in CS consists mainly of Python, Scheme, MATLAB, Mathematica and C++.
       However, my non-formal experience greatly supersedes this, see
       previous employments and further details under the section below regarding programming.}

\entry{Naturvetenskapligt program, Internationella Kunskapsgymnasiet}
      {2008 -- 2011}
      {\textit{2600 points, grade average: 22.0 (of 22.5)}. Specialization courses in physics, mathematics,
       biotechnology and chemistry. The majority of courses where given in English.}
