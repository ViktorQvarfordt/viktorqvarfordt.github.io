\area{IT \& programmering} % IT \& programming skills


\label{programming}


\newenvironment{myitemize}{%
  \let\olditem\item
  \renewcommand\item[1][]{\olditem \textbf{##1}\hspace{0.35em}}
  \begin{list}{}{%
    \setlength{\leftmargin}{1.5em}
    \setlength{\itemsep}{0.25em}
    \setlength{\parskip}{0em}
    \setlength{\parsep}{0.25em}
  }
}{%
  \end{list}
}

\begin{myitemize}
  \item[Operativsystem]
    Mycket god vana i GNU/Linux, som jag använder för alla mina
    ändamål. Använder fritt terminalen och vanliga unix-verktyg som
    sed, grep, ssh, etc. (Emacs)

  \item[Programmering]
    \emph{Python} -- det språk jag är mest bekväm i, har bland annat
    programmerat simuleringar av klassisk mekanik med pygame som
    grafik-API.
    \emph{Javascript} -- webapplikationer, se nedan.
    \emph{Bash} -- enklare script och automatiserade processer, jag
    har bl.a.\ annat har skrivit ett automatiserat backup-system
    (\url{github.com/c3l/ibr}).
    \emph{Övrigt} -- Scheme och Emacs Lisp, bekväm med reguljära uttryck.

  \item[Webbutveckling]
    \begin{list}{}{%
      \setlength{\leftmargin}{1.5em}
      \setlength{\itemsep}{0em}
      \setlength{\parskip}{0em}
      \setlength{\parsep}{0.25em}
    }
      \item[Webbläsare]
        \emph{Backbone.js} -- model-view-controller disignmönster.
        \emph{jQuery} -- kraftfullare DOM-manipulationer.
        \emph{HTML5} -- teknologier så som canvas elementet, web workers och websockets.
        \emph{Server$\,\leftrightarrow\,$klient} -- JSON över AJAX anrop och/eller websockets
        (gärna Socket.IO) beroende på hur dataintensiv applikationen är.
      \item[Server och databas]
        Jag använder ofta \emph{node.js} med \emph{mongodb} som
        databas för att tillhandahålla ett REST API genom vilket
        klienten utför CRUD operationer. Jag har även gjort liknande
        system Python (Django och Flask) och SQlite som databas, jag
        planerar även att använda Ruby on Rails. För driftsättning av
        servermjukvara använder jag heroku som `cloud platform as a
        service`.
    \end{list}

    Exempel som använder flera av dessa teknologier är min
    hemsida \url{nvmq.se}.

  \item[Numerisk/Data-analys] I min utbildning ingår mycket dataanalys och
    tillämpning av numeriska metoder för lösning av komplexa problem.  Jag
    använder främst MATLAB och skriver även objektorienterad
    MATLAB-kod. Jag har även erfarenhet av Mathematica och SciPy.

    Jag har erfarenhet av att arbeta med statistiska och experimentella metoder för
    analys av större mängder data från experiment.
  \item[Övrigt] \emph{Versionshantering:} Git,
    Mercurial. \,\, \emph{Digital typsättning:} \LaTeX
\end{myitemize}


%% \begin{itemize}
%% \item one \dots{}
%%      \begin{itemize}
%%         \item Language Models
%%         \item Vector Space Models
%%      \end{itemize}
%% \item two \dots{}
%% \item three \dots{}
%% \end{itemize}
